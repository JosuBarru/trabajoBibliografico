% This is samplepaper.tex, a sample chapter demonstrating the
% LLNCS macro package for Springer Computer Science proceedings;
% Version 2.20 of 2017/10/04
%
\documentclass[runningheads]{llncs}
%
\usepackage{graphicx}

%Paquetes añadidos
\usepackage{float}
\usepackage{subcaption}
\captionsetup{compatibility=false}

% Used for displaying a sample figure. If possible, figure files should
% be included in EPS format.
%
% If you use the hyperref package, please uncomment the following line
% to display URLs in blue roman font according to Springer's eBook style:
% \renewcommand\UrlFont{\color{blue}\rmfamily}

\begin{document}
	%
	\title{Bibliographic work\thanks{Supported by organization x.}}
	%
	%\titlerunning{Abbreviated paper title}
	% If the paper title is too long for the running head, you can set
	% an abbreviated paper title here
	%
	\author{Erik Cembreros\inst{1}\orcidID{0000-1111-2222-3333} \and
		Josu Barrutia\inst{2,3}\orcidID{1111-2222-3333-4444} \and
		Third Author\inst{3}\orcidID{2222--3333-4444-5555}}
	%
	\authorrunning{F. Author et al.}
	% First names are abbreviated in the running head.
	% If there are more than two authors, 'et al.' is used.
	%
	\institute{Princeton University, Princeton NJ 08544, USA \and
		Springer Heidelberg, Tiergartenstr. 17, 69121 Heidelberg, Germany
		\email{lncs@springer.com}\\
		\url{http://www.springer.com/gp/computer-science/lncs} \and
		ABC Institute, Rupert-Karls-University Heidelberg, Heidelberg, Germany\\
		\email{\{abc,lncs\}@uni-heidelberg.de}}
	%
	\maketitle              % typeset the header of the contribution
	%
	\begin{abstract}
			
		
		\keywords{First keyword  \and Second keyword \and Another keyword.}
	\end{abstract}
	%
	%
	%
	
	\clearpage
	\section{Results analysis}
	
	Five prespecified access point topologies have been considered to compare how do they affect the placement and movement trends of the jammers. To achieve this the simplified bilevel mixed-
	integer program [BLMIP] has been used.
	
	The Partite topology had three clusters with demand points clustering around them. The Perimeter topology distributed access points along the region's perimeter, allowing free movement of demand points. The Dense topology featured a central hub with clustered access points, catering to high demand independently. The Spacious topology randomly distributed access points to avoid proximity, resulting in demand points within range of one or two access points. The Median topology had uniformly distributed access points along diagonals and central lines, with clustering in the center, resembling a campus with a central area of high connectivity demand.
	

	For all five topologies, three experiments with 10, 25 and 50 AP, each with a capacity of 15, 5 jammers with jamming radius of 150 feet, and 5 time periods. Each region was 1 square mile. The demand was realized ten times, with 100 demand points, and the results were averaged.
	
	It was found that access point placement near concentrations of demand points was crucial for ensuring robust network connectivity. However, access points too clustered together give jammers an easier time on severing connections. Therefore, the optimal solution should find a balance between these two factors. 
	
	Sensitivity has also been proved, concluding that adding a set of new access points did not improve overall connectivity. Meanwhile, were more jammers to be added to the original problem, the set of new APs would become significantly important.
	%%Aqui habria que referenciar la ecuacion 3a que es la utilizada para comparar las topologias 
	\ref{Eq3a} is a generalized utility function considering three aspects: range, unity and tolerance.
	The range describes signal strength for access-demand point connections, unity represents uniform signal strength with either connected or unconnected links and tolerance introduces a threshold below which connections are considered too weak to be useful and are treated as unconnected.
	
	
	
	\section{Conclusions}
	
	The optimal placement of access points is crucial to ensure maximum connectivity for demand points in various environments, including university campuses. This optimal topology will also maximize connectivity in the presence of jammers, considering both non-additive and additive models.
	Achieving a proper placement requires striking a balance between placing access points near concentrations of demand points and avoiding the formation of clusters.	In this regard, the Partite and Median topologies demonstrated greater robustness against jamming attacks when considering utility as total signal strength, number of connections, or tolerance allowance.
	
	While this research considers the placement of both access points and jammers, the placement of demand points has not been taken into account. Considering this would lead to a stochastic problem rather than a deterministic one.
	
	
	%
	% ---- Bibliography ----
	%
	% BibTeX users should specify bibliography style 'splncs04'.
	% References will then be sorted and formatted in the correct style.
	%
	% \bibliographystyle{splncs04}
	% \bibliography{mybibliography}
	%
	\begin{thebibliography}{8}
		\bibitem{ref_article1}
		Author, F.: Article title. Journal \textbf{2}(5), 99--110 (2016)
		
	\end{thebibliography}
\end{document}

